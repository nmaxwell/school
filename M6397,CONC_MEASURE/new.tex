\documentclass[12pt]{article}
\usepackage{geometry,amsthm,amsmath, amscd,amssymb, graphicx, natbib, float, enumerate}
\geometry{margin=1in}
\renewcommand{\familydefault}{cmss}
%\usepackage{charter}
\restylefloat{table}
\restylefloat{figure}

%%%%%%%%%%%% MODIFY LECTURE DATE AND AUTHOR %%%%%%%%%%%%%%%%%

\newcommand\lecdat{Jan 26, 2010% INSERT LECTURE DATE HERE
}
\newcommand\notesby{Nick Maxwell% INSERT NOTE TAKER HERE
}

%%%%%%%%%%%%%%%%%%%%%%%%%%%%%%%%%%%%%%%%%%%%%%%%%%%

\date{} % not needed
\author{} % not needed

%%%%%%%%%%% SOME MACROS BELOW %%%%%%%%%%%%%%%%%%%%%%%%%%

\swapnumbers
\newtheorem{thm}{Theorem}[section]
\newtheorem{claim}[thm]{Claim}
\newtheorem{cor}[thm]{Corollary}
\newtheorem{conclusion}{Conclusion}

\newtheorem{lemma}[thm]{Lemma}
\newtheorem{prop}[thm]{Proposition}
\newtheorem{defn}[thm]{Definition}
\theoremstyle{remark}
\newtheorem{rem}[thm]{Remark}
\newtheorem{remarks}[thm]{Remarks}
%\newtheorem{rem}[thm]{Remark}
\newtheorem{ex}[thm]{Example}
\newtheorem{exc}[thm]{Exercise}
%\newtheorem{fact}[thm]{Fact}
\newtheorem{facts}[thm]{Facts}
\newtheorem{prob}[thm]{Problem}
\newtheorem{question}[thm]{Question}
\newtheorem{answer}[thm]{Answer}
\newtheorem{conj}[thm]{Conjecture}

\renewcommand{\thethm}{\thesubsection.\arabic{thm}}

\newcommand{\bb}[1]{\mathbb{#1}}
\newcommand{\cl}[1]{\mathcal{#1}}
\newcommand{\ff}[1]{\mathfrak{#1}}

\newcommand{\norm}[1]{\|#1\|}
\newcommand{\abs}[1]{|#1|}
\def\eps{\epsilon}
\def\del{\delta}

\def\Sn{\mathbb S^n}
\def\Snm1{\mathbb S^{n-1}}

\def\R{\mathbb R}
\def\Rn{\mathbb R^n}

%%%%%%%%%%% PUT YOUR MACROS HERE %%%%%%%%%%%%%%%%%%%%%%%%

\newcommand{\pset}[1]{ \mathcal{P}(#1) }

\newcommand{\nats}[0] { \mathbb{N}}
\newcommand{\reals}[0] { \mathbb{R}}
\newcommand{\exreals}[0] {  [-\infty,\infty] }
%\newcommand{\eps}[0] {  \epsilon }
\newcommand{\A}[0] { \mathcal{A} }
\newcommand{\B}[0] { \mathcal{B} }
\newcommand{\C}[0] { \mathcal{C} }
\newcommand{\D}[0] { \mathcal{D} }
\newcommand{\E}[0] { \mathcal{E} }
\newcommand{\F}[0] { \mathcal{F} }
\newcommand{\G}[0] { \mathcal{G} }
\newcommand{\cO}[0] { \mathcal{O} } % curly O

\newcommand{\om}[0] { \omega }
\newcommand{\Om}[0] { \Omega }

\newcommand{\Bl}[0] { \mathcal{B} \ell } %borel

\newcommand{\st}[0]{ \; \textrm{s.t.} \; } 

\newcommand{\IF}[0] { \; \textrm{if} \; }
\newcommand{\THEN}[0] { \; \textrm{then} \; }
\newcommand{\ELSE}[0] { \; \textrm{else} \; }
\newcommand{\AND}[0]{ \; \textrm{ and } \;  }
\newcommand{\OR}[0]{ \; \textrm{ or } \; }

\newcommand{\rimply}[0] { \Rightarrow }
\newcommand{\limply}[0] { \Lefttarrow }
\newcommand{\rlimply}[0] { \Leftrightarrow }
\newcommand{\lrimply}[0] { \Leftrightarrow }

\newcommand{\rarw}[0] { \rightarrow }
\newcommand{\larw}[0] { \leftarrow }

%%%%%%%%%%%%%%%%%%%%%%%%%%%%%%%%%%%%%%%%%%%%%%%%%%%

\title{{\bf High-Dimensional Measures and Geometry}\\
Lecture Notes from \lecdat\\[0.1cm] \small taken by \notesby}

\begin{document}
\maketitle

%\setcounter{section}{-1}


Denote by $G_k( \reals^n )$ the Grassmanian, which is the collection of $k$-dimensional subspaces of $\reals^n$.
Define a distance on $G_k( \reals^n )$ by the operator norm of the difference between corresponding orthogonal projections.
That is, $P_1 : \reals^n \rarw V_1$, $P_2 : \reals^n \rarw V_2$, with $V_1, V_2$ k-dimensional, then $d(V_1, V_2) = ||P1-P2||$.
This distance is invariant under the orthogonal group. So, $||P_1-P_2|| = ||  O P_1 O^* - O P_2 O^* || = ||  O (P_1 - P_2) O^* ||$, $O \in \cO(n)$, the set of unitary operators on $\reals^n$.

Also, $\cO(n)$ acts transitively on projections, for all rank-$k$ $P_1, P_2$, $\exists O \in \cO(n) \st P_2 = O P_1 O^*$ $\rimply \exists !$ Borel probability measure on $G_k( \reals^n )$, invaraint under $\cO(n)$, we denote this measure by $\mu_{n,k}$.

This measure can be obtained from the left-invariant haar measure $\nu_n$ on $\cO(n)$ by the map

$$
    \Psi: \cO \rarw O P_{V_1} O^*
$$

$P_{V_1}$ an orthogonal projection onto some fixed $k$-dimensional substapce.

In terms of subspaces, we have 

$$
	\mu_{n,k}(V) = \nu_n ( \{  U \in \cO(n): U(V_1) \in V \} ), V \in G_k( \reals^n )
$$

Why is this identity true? This is because the image measure is invariant under the action of $\cO(n)$, by the commutative diagram below.




\begin{equation}
\begin{CD}
\cO(n) @>{ O \mapsto O'O \; \; ^{[1]}  }>> \cO(n) \\
@VV{ \Psi }V         @VV{\Psi  }V \\
G_k(\reals^n) @>{ V \mapsto O' V  \; \; ^{[1]} }>> G_k(\reals^n)
\end{CD}
\end{equation}

[1] This is left mupltiplication by $O'$, some fixed $O' \in \cO(n)$\\

The ``effective map'' between $G_k( \reals^n)$ is because $ O'O P_{V_1} O^*(O')^*$ and this projection has range $ O'(O(V_1)) = (O'O)(V_1)$



\begin{lemma}

Let $x \in \reals^n, x \not = 0$, let $\mu_{n,k}$ be the $\cO(n)$-invariant measure on $G_k(\reals^n)$, and for each $V\in G_k(\reals^n)$, let $P_V$ denote orthogonal projection onto $V$. Then, for $0 < \eps < 1$,

$$
	\mu_{n,k} ( \{ V \in G_k( \reals^n) ; \sqrt{\frac{n}{k}}  || P_V(x)|| \ge \frac{1}{1- \eps} ||x||  \} )   \le  \exp{( - \eps^2 k /4 )} + \exp{( - \eps^2 n /4 )}
$$

and 

$$
	\mu_{n,k} ( \{ V \in G_k( \reals^n) ; \sqrt{\frac{n}{k}}  || P_V(x)|| \le (1- \eps) ||x||  \} )   \le  \exp{( - \eps^2 k /4 )} + \exp{( - \eps^2 n /4 )}
$$

\begin{proof} 

	Without loss of generality, choose $||x||=1$, choose any $k$-dimensional subspace, $V_1$, and if $U \in \cO(n)$, let $V=U(V_1)$, $P_V$ the orthogonal projection onto $V_1$, and use the fact that the measure $\nu_n$ on $\cO(n)$ induces the Grassmanian measure $\nu_{n,k}$.
	
	This implies, 
	
$$
	\mu_{n,k} ( \{ V \in G_k( \reals^n) ; \sqrt{\frac{n}{k}}  || P_V(x)|| \ge \frac{1}{1- \eps}   \} )   = \nu_n( \{  U \in \cO(n) ; \sqrt{\frac{n}{k}} || P_{U(V_1)} (x) || \ge \frac{1}{1-\eps} \} )
$$
	and
	
$$
	\mu_{n,k} ( \{ V \in G_k( \reals^n) ; \sqrt{\frac{n}{k}}  || P_V(x)|| \le (1- \eps)  \} )   = \nu_n( \{  U \in \cO(n) ; \sqrt{\frac{n}{k}} || P_{U(V_1)} (x) || \le \frac{1}{1-\eps} \} )
$$

The projected length of $x$ is

$$
	||P_{U(V_1)}(x)|| = || U^* P_{U(V_1)} U U^* x || = || P_{V_1} U^* x ||
$$

and the image measure induced by $\nu_n$ under $\Phi_x: \cO(n) \rarw S^{n-1} U \mapsto U^*x$ is the surface measure on sphere, $\mu_n$.

Thus,

$$
	\nu_n( \{  U \in \cO(n) ; \sqrt{\frac{n}{k}} || P_{U(V_1)} (x) || \ge \frac{1}{1-\eps} \} ) = \mu_{n} ( \{ y \in S^{n-1} ; \sqrt{\frac{n}{k}}  || P_V(y)|| \ge \frac{1}{1- \eps}   \} ) 
$$
	and
	
$$
	\nu_n( \{  U \in \cO(n) ; \sqrt{\frac{n}{k}} || P_{U(V_1)} (x) || \le (1-\eps) \} ) = \mu_{n} ( \{ y \in S^{n-1} ; \sqrt{\frac{n}{k}}  || P_V(y)|| \le (1- \eps)  \} ) 
$$

now applyying the corollary in section 2.3 (Gaussian v.s. surface measure), finishes the proof.

\end{proof}

\end{lemma}

Summary: Now reduction for vectors ib $S_{n-1}$ under fixed program is ``mostly'' by factor $\sqrt{\frac{k}{n}}(1 \pm \eps )$, same is true for fixed vector under projections onto ``many subspaces'', in $G_k(\reals^n)$.

Question: what about more than one vector?

\begin{thm}

(Johnson-Lindenstrauss, Part II)

Let $a_1, ..., a_N$ be points in $\reals^n$, given $\eps>0$, choose $k \in \nats \st$

$$
	N(N+1) ( \exp(-k \eps^2/4) + \exp(-n \eps^2/4) ) \le \frac{1}{3}
$$

and let $G_k( \reals^n )$ be the set of $k$-dimensional subspaces, then

$$
	\mu_{n,k} (\{ V \in G_k( \reals^n ) ; (1-\eps) || a_i - a_j|| \le \sqrt{\frac{k}{n}} || P_V(a_i - a_j)|| \le \frac{1}{1-\eps} || a_i - a_j||  \; \forall \; 1 \le i \le j \le N \}) \ge \frac{2}{3}
$$

proof:

Let $c_{ij} = a_i-a_j, i>j$, we count $N \choose 2 = N(N-1)/2$ such differences, and $ ||P_V c_{ij}|| = ||P_V a_i - P_V a_j ||$ the set of subspaces $V$ for which $\sqrt{\frac{k}{n}} || P_V c_{ij} || \ge \frac{1}{1-\eps} ||c_{ij}||$ or $\sqrt{\frac{k}{n}} || P_V c_{ij} || \le (1-\eps) ||c_{ij}||$ has by assumption and union bound over choices $i,j \in \{ 1,2,...,N\}, i \not = j$ measure at most $\frac{1}{3}$.

Thus by taking the complement, gives the desired estimate of thge measure.

\end{thm}

Question:

What about infinitely many vector, i.e. $ \textrm{span} \{ a_1, ..., a_T \}$, for some $T \in \nats$ ? ``restriced isometry property''

Need to choose set of points $Q \subset \{ x \in \textrm{span} \{ a_1, ..., a_T \};  ||x|| =1 \}$  ``sufficiently dense '', apply Johnson-Lindenstrauss to $Q$, combine this with triangle inequality to get estimate for all points. 


\end{document}
