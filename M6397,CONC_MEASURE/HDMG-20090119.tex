\documentclass[12pt]{article}
\usepackage{geometry,amsthm,amsmath,amssymb, graphicx, natbib, float, enumerate}
\geometry{margin=1in}
\renewcommand{\familydefault}{cmss}
%\usepackage{charter}
\restylefloat{table}
\restylefloat{figure}

%%%%%%%%%%%% MODIFY LECTURE DATE AND AUTHOR %%%%%%%%%%%%%%%%%

\newcommand\lecdat{Jan 19, 2010% INSERT LECTURE DATE HERE
}
\newcommand\notesby{Bernhard Bodmann% INSERT NOTE TAKER HERE
}

%%%%%%%%%%%%%%%%%%%%%%%%%%%%%%%%%%%%%%%%%%%%%%%%%%%

\date{} % not needed
\author{} % not needed

%%%%%%%%%%% SOME MACROS BELOW %%%%%%%%%%%%%%%%%%%%%%%%%%

\swapnumbers
\newtheorem{thm}{Theorem}[section]
\newtheorem{claim}[thm]{Claim}
\newtheorem{cor}[thm]{Corollary}
\newtheorem{conclusion}{Conclusion}

\newtheorem{lemma}[thm]{Lemma}
\newtheorem{prop}[thm]{Proposition}
\newtheorem{defn}[thm]{Definition}
\theoremstyle{remark}
\newtheorem{rem}[thm]{Remark}
\newtheorem{remarks}[thm]{Remarks}
%\newtheorem{rem}[thm]{Remark}
\newtheorem{ex}[thm]{Example}
\newtheorem{exc}[thm]{Exercise}
%\newtheorem{fact}[thm]{Fact}
\newtheorem{facts}[thm]{Facts}
\newtheorem{prob}[thm]{Problem}
\newtheorem{question}[thm]{Question}
\newtheorem{answer}[thm]{Answer}
\newtheorem{conj}[thm]{Conjecture}

\renewcommand{\thethm}{\thesubsection.\arabic{thm}}

\newcommand{\bb}[1]{\mathbb{#1}}
\newcommand{\cl}[1]{\mathcal{#1}}
\newcommand{\ff}[1]{\mathfrak{#1}}

\newcommand{\norm}[1]{\|#1\|}
\newcommand{\abs}[1]{|#1|}
\def\eps{\epsilon}
\def\del{\delta}

\def\Sn{\mathbb S^n}
\def\Snm1{\mathbb S^{n-1}}

\def\R{\mathbb R}
\def\Rn{\mathbb R^n}

%%%%%%%%%%% PUT YOUR MACROS HERE %%%%%%%%%%%%%%%%%%%%%%%%

\newcommand\mynewcommand{use often}

%%%%%%%%%%%%%%%%%%%%%%%%%%%%%%%%%%%%%%%%%%%%%%%%%%%

\title{{\bf High-Dimensional Measures and Geometry}\\
Lecture Notes from \lecdat\\[0.1cm] \small taken by \notesby}

\begin{document}
\maketitle

\setcounter{section}{-1}

\section{Course Information}

\begin{description}
\item[Text:] Michel Ledoux, The Concentration of Measure Phenomenon, American Mathematical Society, Providence, Rhode Island, 2001.
\item[Office:] Tu 1-2pm, Th 11-12 or by appointment
\item[Email:] bgb@math.uh.edu
\item[Grade:] Based on preparation of class notes in LaTeX, rotating note-takers 
\end{description}

\section{Introduction}

Informally, the phenomenon of measure concentration is that in a given ``high-dimensional'' probability space, any
``well-behaved'' function is ``almost'' constant. We will study how to make this statement quantitative and precise and what consequences can be derived from this measure concentration. To get a taste of this subject, we start with a simple example,
concentration on high-dimensional spheres.

\subsection{Measure concentration on high-dimensional spheres}

Consider the {\em unit sphere} $\Snm1 = \{x \in \Rn: \|x\|=1\}$, containing each vector $x \in \mathbb R^n$
with {\em Euclidean norm} $\|x\|=(\sum_{i=1}^n x_i^2 )^{1/2}=1$. We also define the {\em canonical inner
product} between two vectors $x, y \in \Rn$ by $\langle x,y\rangle =\sum_{i=1}^n x_i y_i$. Define the 
{\em metric distance} between $x,y \in \Snm1$ by
$$
    d(x,y) = \arccos \langle x, y \rangle \in [0,\pi] \, .
$$

\begin{question} Why is the function $d: \Snm1 \times \Snm1 \to \mathbb R^+$ a metric?
\end{question}

Symmetry and positive definiteness are elementary.
The distance $d(x,y)$ is the length of the shortest geodesic (great arc) between $x$ and $y$. A direct proof of
the triangle inequality can be found in M. Berger, Geometry II, Springer, N.Y., 1996. Otherwise, we can 
appeal to the fact that the shortest geodesic between two points minimizes length among all piecewise differentiable
paths, and thus if the path had to pass through another point in between, then the length can only increase.

We also define the {\em rotation-invariant probability measure} $\mu$ on $\Snm1$. This is induced on the sphere by 
the (left) Haar measure on the Lie group of all orientation-preserving rotations. For a given continuous function
$f: \Snm1 \to \R$, we define the {\em median} $m_f$ with respect to $\mu$ by the two properties
$$
   \mu(\{x \in \Snm1: f(x) \ge m_f\} \ge \frac 1 2
$$ 
and
$$
    \mu(\{x \in \Snm1: f(x) \le m_f\} \ge \frac 1 2 \, .
$$

\begin{question} Why is this median well defined?\end{question}

For each $\alpha\in \mathbb R$, define $L_\alpha = \{x \in \Snm1: f(x) < \alpha\}$ and
$U_\alpha = \{x \in \Snm1: f(x) \ge \alpha\}$,
then $g(\alpha)=\mu(L_\alpha)$ and $h(\alpha)=\mu(U_\alpha)$ are by definition monotonic in $\alpha$ and
$g+h=1$. Let the median $m_f$
be given by $m_f=\inf\{ \alpha: g(\alpha)\ge 1/2\}$, then it satisfies the second of the two claimed inequalities
by the regularity of $\mu$. Moreover, we also have $m_f = \sup\{ \alpha:  h(\alpha) \ge 1/2\}$, otherwise we would
get a contradiction to $g+h=1$. Using the regularity of $\mu$, the first inequality follows.

\begin{defn} A function $ f : \Snm1 \to \R$ is $M$-Lipshitz if
$$
    |f(x) - f(y)| \le M d(x,y)
$$
for all $x, y \in \Snm1$.
\end{defn}

We can now state a first result in the context of  measure concentration.

\begin{prop} \label{prop:consnm1}Let $\Snm1$ be the unit sphere in $\mathbb R^n$ and $\mu$ the
rotation-invariant probability measure on $\Snm1$. If $f: \Snm1 \to \R$ is 1-Lipshitz and $m_f$ is its
median with respect to $\mu$, then for $\eps\ge 0$,
$$
    \mu(\{x \in \Snm1: |f(x) - m_f| \le \eps\}) \ge 1 - \sqrt{\frac{\pi}{2}} e^{- \eps^2 (n-2)/2} \, . 
$$
\end{prop}
\begin{proof} Later. \end{proof}

Note, the inequality gives no information for $\eps\to 0$ when $n$ is held fixed. However, if we choose a sequence $\{\eps_n\}_{n \in \mathbb N}$ and let $\epsilon_n \to 0$ such that $\eps_n^2 n \to \infty$, then the measure 
$\mu(\{x \in \Snm1: |f(x) - m_f| \le \eps\}) \to 1 .$

We will consider an example of a function $f$ which is natural for the metric. 

\begin{question} Why is the function the function $f(x)=\langle x, a\rangle$,
for $a \in \Snm1$,
 $1$-Lipshitz?
\end{question}
We have
$$
    |f(x)-f(y)| = |\langle x- y, a\rangle | \le \|x-y\| \|a \| = \|x-y\| \le d(x,y) \, , 
$$
because $\|x-y\|$ is the chordal distance, the length of the geodesic in $\Rn$ between $x$ and $y$,
whereas $d(x,y)$ measures the length of the shortest geodesic in the subset $\Snm1 \subset \Rn$.

\begin{exc} Derive a concentration inequality of the form stated in \ref{prop:consnm1} in the special case $f(x) = \langle x, a\rangle$
for any $ a \in \Snm1$.
\end{exc}

%\begin{proof}
Without loss of generality, we can choose $a=(1, 0, 0, \dots, 0)$ after a suitable rotation of the sphere.
The median of $f$ is then identified as $m_f=0$, because the ``northern'' hemisphere
$$
  \{ x \in \Snm1: \langle x,a\rangle \ge 0\} =  \{ x \in \Snm1: x_1 \ge 0\}
$$
has measure $1/2$ and so does its reflection about the origin.

Consider the image measure $\rho_n$ on $[-1,1]$ induced by the projection of $\mu$ with $f$.
It is absolutely continuous with respect to the Lebesgue measure and has a Radon-Nikodym derivative
$$
   d\rho_n(t) = A_n (1-t^2)^{(n-3)/2} dt \, ,
$$
with the normalization constant $A_n = \frac{\Gamma(n/2)}{\sqrt \pi \Gamma((n-1)/2)}$.
We compute
$$
    \mu (\{ x \in \Snm1: |\langle x,a\rangle| \le \eps\}) = \rho_n([-\eps,\eps]) \, ,
$$
and use normalization and symmetry
$$
  \rho_n([-\eps,\eps]) = \int_{-\eps}^\eps A_n (1-t^2 )^{(n-3)/2} dt 
  = 1 - 2 \int_\eps^1 A_n (1-t^2)^{(n-3)/2} dt \, . 
$$
The integrand is estimated on the interval $[\eps,1]$ by
$$
   1-t^2 \le e^{- t^2} \le e^{-(n-3)\eps^2/2} e^{-(n-3)(t-\eps)}
$$
where the exponent has been linearized about $t=\eps$.
Further, extending the domain  to $[\eps, \infty)$ (if $n>3$) and performing the integration gives
$$
   \rho_n([-\eps,\eps]) \ge 1 - \frac{2 A_n}{n-3} e^{-(n-3)\eps^2/2}  \, .
$$
This bound is of the same exponential type as in \ref{prop:consnm1}.
Now the constant $2A_n/(n-3)$ can be estimated with the help of Stirling's approximation,
$$
   \frac{2 A_n}{n-3} \approx \sqrt{\frac{2 e}{ n \pi }}
$$
which vanishes as $n \to \infty$. Thus, this estimate is asymptotically slightly better than the result for the general
case in \ref{prop:consnm1}.
%\end{proof}

We summarize this  exercise informally by saying that the surface measure of the unit sphere in $\Rn$ is
concentrated near the equator, or in any set containing the vicinity of a great arc.

As we can see from the exercise, integration on the sphere is not straightforward. Fortunately, 
we can derive many concentration results with a detour through Gaussian measures in $\Rn$.

We will see that the Gaussian measure is in high dimensions close to the surface measure of an appropriately scaled sphere.

\section{Wrapping the sphere in a Gaussian measure}

The standard Gaussian measure $\gamma_1$ on $\mathbb R^1$ has the density
$
   \frac{1}{\sqrt{2\pi}} e^{-t^2/2} \, .
$
It can be verified that this is a probability measure, and so is the standard Gaussian measure $\gamma_n$ on $\Rn$
with density 
$$
  \frac{1}{(2\pi)^{n/2}} e^{-\|x\|/2} \, ,
$$
so for a measurable set $A \subset \Rn$,
$$
  \gamma_n(A) = \int_A  \frac{1}{(2\pi)^{n/2}} e^{-\|x\|/2} dx \, .
$$

We will find that the function $x \mapsto \|x\|$ is almost constant with respect to the Gaussian measure,
with value $\|x\|\approx \sqrt n$.
This means, with respect to the standard Gaussian measure, most vectors are close to the sphere of radius $\sqrt n$. 

\subsection{The Laplace transform}

\begin{question} How do we estimate $\mu(\{x: f(x) \ge a\})$, $a \in \mathbb R$
for some probability space $(X,\mathcal A, \mu)$?
\end{question}

We use a trick: Let $\lambda>0$, then 
$$
   f(x) \ge a \Leftrightarrow \lambda f \ge \lambda a \Leftrightarrow e^{\lambda f} \ge e^{\lambda a}\, .
$$
By positivity of the exponential function,
$$
   \int_X e^{\lambda f} d\mu \ge \int_{x\in X: f(x)\ge a} e^{\lambda f} d\mu
$$
and then using the lower bound for the exponential gives
$$
   \int_X e^{\lambda f} d\mu \ge \mu(\{x \in X: f(x) \ge a\}) e^{\lambda a} \, .
$$
Rearranging terms gives
$$
    \mu(\{x \in X: f(x) \ge a\}) \le e^{-\lambda a} \int_X e^{\lambda f} d\mu \, .
$$
The right hand side can often be computed for typical measures and functions. Finally,
$\lambda$ can be chosen such that the right hand side becomes smallest. 

Similarly,
$$
    \mu(\{x \in X: f(x) \le a\}) \le e^{\lambda a} \int_X e^{-\lambda f} d\mu \, .
$$

\end{document}