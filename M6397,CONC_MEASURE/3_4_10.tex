    \documentclass[12pt]{article}
\usepackage{geometry,amsthm,amsmath, amscd,amssymb, graphicx, natbib, float, enumerate}
\geometry{margin=1in}
\renewcommand{\familydefault}{cmss}
%\usepackage{charter}
\restylefloat{table}
\restylefloat{figure}

%%%%%%%%%%%% MODIFY LECTURE DATE AND AUTHOR %%%%%%%%%%%%%%%%%

\newcommand\lecdat{Mar 2, 2010% INSERT LECTURE DATE HERE
}
\newcommand\notesby{Nick Maxwell% INSERT NOTE TAKER HERE
}

%%%%%%%%%%%%%%%%%%%%%%%%%%%%%%%%%%%%%%%%%%%%%%%%%%%

\date{} % not needed
\author{} % not needed

%%%%%%%%%%% SOME MACROS BELOW %%%%%%%%%%%%%%%%%%%%%%%%%%

\swapnumbers
\newtheorem{thm}{Theorem}[section]
\newtheorem{claim}[thm]{Claim}
\newtheorem{cor}[thm]{Corollary}
\newtheorem{conclusion}{Conclusion}

\newtheorem{lemma}[thm]{Lemma}
\newtheorem{prop}[thm]{Proposition}
\newtheorem{defn}[thm]{Definition}
\theoremstyle{remark}
\newtheorem{rem}[thm]{Remark}
\newtheorem{remarks}[thm]{Remarks}
%\newtheorem{rem}[thm]{Remark}
\newtheorem{ex}[thm]{Example}
\newtheorem{exc}[thm]{Exercise}
%\newtheorem{fact}[thm]{Fact}
\newtheorem{facts}[thm]{Facts}
\newtheorem{prob}[thm]{Problem}
\newtheorem{question}[thm]{Question}
\newtheorem{answer}[thm]{Answer}
\newtheorem{conj}[thm]{Conjecture}
\newtheorem{idea}[thm]{Idea}

\renewcommand{\thethm}{\thesubsection.\arabic{thm}}

\newcommand{\bb}[1]{\mathbb{#1}}
\newcommand{\cl}[1]{\mathcal{#1}}
\newcommand{\ff}[1]{\mathfrak{#1}}

\newcommand{\norm}[1]{\|#1\|}
\newcommand{\abs}[1]{|#1|}
\def\eps{\epsilon}
\def\del{\delta}

\def\Sn{\mathbb S^n}
\def\Snm1{\mathbb S^{n-1}}

\def\R{\mathbb R}
\def\Rn{\mathbb R^n}

%%%%%%%%%%% PUT YOUR MACROS HERE %%%%%%%%%%%%%%%%%%%%%%%%

\newcommand{\pset}[1]{ \mathcal{P}(#1) }

\newcommand{\nats}[0] { \mathbb{N}}
\newcommand{\reals}[0] { \mathbb{R}}
\newcommand{\exreals}[0] {  [-\infty,\infty] }
%\newcommand{\eps}[0] {  \epsilon }
\newcommand{\A}[0] { \mathcal{A} }
\newcommand{\B}[0] { \mathcal{B} }
\newcommand{\C}[0] { \mathcal{C} }
\newcommand{\D}[0] { \mathcal{D} }
\newcommand{\E}[0] { \mathcal{E} }
\newcommand{\F}[0] { \mathcal{F} }
\newcommand{\G}[0] { \mathcal{G} }
\newcommand{\cO}[0] { \mathcal{O} } % curly O

\newcommand{\om}[0] { \omega }
\newcommand{\Om}[0] { \Omega }

\newcommand{\Bl}[0] { \mathcal{B} \ell } %borel

\newcommand{\st}[0]{ \; \textrm{s.t.} \; } 

\newcommand{\IF}[0] { \; \textrm{if} \; }
\newcommand{\THEN}[0] { \; \textrm{then} \; }
\newcommand{\ELSE}[0] { \; \textrm{else} \; }
\newcommand{\AND}[0]{ \; \textrm{ and } \;  }
\newcommand{\OR}[0]{ \; \textrm{ or } \; }

\newcommand{\rimply}[0] { \Rightarrow }
\newcommand{\limply}[0] { \Lefttarrow }
\newcommand{\rlimply}[0] { \Leftrightarrow }
\newcommand{\lrimply}[0] { \Leftrightarrow }

\newcommand{\rarw}[0] { \rightarrow }
\newcommand{\larw}[0] { \leftarrow }

\newcommand{ \defeq }[0] { \colonequals }
\newcommand{ \eqdef }[0] { \equalscolon }

\usepackage{colonequals}

\newcommand{ \Ex }[1] { E\left[ #1 \right ] }

%%%%%%%%%%%%%%%%%%%%%%%%%%%%%%%%%%%%%%%%%%%%%%%%%%%

\title{{\bf High-Dimensional Measures and Geometry}\\
Lecture Notes from \lecdat\\[0.1cm] \small taken by \notesby}

\begin{document}
\maketitle

%\setcounter{section}{-1}

\begin{proof} 

Consider the extention, $g$, of $f$ to all of $\reals^n$, Let $G(x,y) = g(x) - g(y)$, then

$$
    E_{\mu_n \times \mu_n } \left[  e^{ f(x) - f(y) }   \right] = E_{ \tilde{\gamma}_{2n} } \left[  e^{G(x,y)}   \right],
$$

\noindent
$\tilde{\gamma}_{2n} $ with density $ \frac{1}{\left( 2 \pi \sigma^2 \right)^n} e^{\frac{-||x||^2}{2 \sigma^2}} $, where $\sigma$ is chosen appropriately, `(so sum of random variables variance = 1)'. Then,

$$
 E_{ \tilde{\gamma}_{2n}} \left[  e^{G(x,y)}  \right] \le E_{ \tilde{\gamma}_{2n} } \left[ e^{ \frac{ \pi^2} { 8 } } || \nabla f||^2 \sigma^2  \right]
$$

\noindent
so, if $|| \nabla g|| \le 3$, then we conclude that

$$
E_{\mu_n} \left[ e^{\lambda f} \right] \le e^{ 9 \pi^2 (\lambda \sigma ) ^2 / 8 },
$$
 
 \noindent
 we know that as $n \rarw \infty$, $n \sigma^2 \rarw 1$. Using the Laplace transform method,
 
 $$
 \mu_n \left( \{ x\in S^{n-1}; f(x) \ge t \}   \right) \le e^{-\lambda t} \Ex{ e^{\lambda f}} \le e^{- t \lambda } \exp( 9 \pi^2 (\lambda \sigma) ^2 /8) \rimply 
 $$

$$
 \mu \left( \{ x \in S^{n-1}; f(x) \ge t \} \right) \le e^{ \frac{-2t^2}{9 \pi^2 \sigma^2}}
$$
 
\end{proof}


\begin{question}
How much smaller is the set of differentiable functions with $|| \nabla f||^2 \le 1$ compared to 1-Lipschitz functions?
\end{question}

\begin{answer}
Smaller by ``$\epsilon$''. Prove that any 1-Lipschitx function can be approximated uniformly by differentiable functions.
\end{answer}

\begin{thm}
Let $f: \reals^n \rarw \reals$ be 1-Lipschitz, define the localized averages $f(x) = \frac{1}{|B_\epsilon|(x)} \int_{B_\epsilon|(x)} g(y) \, dy$, where $B_\epsilon|(x) = \{ y \in \reals^n; ||x-y \le \epsilon\}$. Then $f$ is differentiable, and for all $x \in \reals$, $|f(x) \ g(x)| \le \frac{ \epsilon n }{n+1} \le \epsilon$, and $|| \nabla f|| = 1$.

\end{thm}

\begin{proof}
First, check $n=1$, then

$$
f(x) = \frac{1}{2 \epsilon} \int_{x-\epsilon}^{x+\epsilon} g(y) \, dy 
$$

\noindent
and by the fundamental theorem of calculus, $f$ is differentiable:

$$
f'(x) = \frac{1}{2 \epsilon} \left[ g(x+ \epsilon) - g(x - \epsilon) \right], |f'(x)| = \frac{1}{2 \epsilon} | g(x+ \epsilon) - g(x - \epsilon) | \le \frac{1}{2 \epsilon} (2 \epsilon) = 1.
$$

\noindent
Moreover,

$$
| f(c) - g(x)| = | g(x) -   \frac{1}{2 \epsilon} \int_{x-\epsilon}^{x+\epsilon} g(y) \, dy | =  | \frac{1}{2 \epsilon} \int_{x-\epsilon}^{x+\epsilon} (g(x) - g(y)) \, dy | \le 
\frac{1}{2 \epsilon} \int_{x-\epsilon}^{x+\epsilon} |g(x) - g(y)| \, dy  = \frac{ \epsilon^2}{ 2 \epsilon} = \frac{\epsilon}{2} .
$$

In higher dimensions, similar analysis works. We only prove that $D_u f(x) = \nabla f(c) \cdot u, ||u|| = 1$ gives $|D_u f(x| \le 1$. Without loss of generality, $x=0$, $u = (1,0,...,0)$.  Have $D_u  f(0) = \frac{d}{dt} f(t u)|_{t=0}$. Define the disk $D_\epsilon = \{  x \in \reals^n; x \perp u, ||x|| \le \eps \}$. We compute $f(x)$ in cylindrical coordinates,

$$
    f(tu) = \frac{1}{|B_\epsilon|} \int_{B_\epsilon + t_u} g(z) \, dz = \frac{1}{|B_\epsilon|} \int_{D_\epsilon} \int_{t - \sqrt{\epsilon^2 - ||y||^2}}^{t + \sqrt{\epsilon^2 - ||y||^2}} g(y+s e_1) \, ds \, dy,
$$

\noindent
where $e_1 = (1,0, 0,.)$, so 

$$
\frac{d}{dt} f(tu) |_{t=0} = \frac{1}{|B_\epsilon|} \left| \int_{D_\epsilon} \left( g(y +  \sqrt{\epsilon^2 + ||y||^2}e_1 ) - g(y +  \sqrt{\epsilon^2 - ||y||^2}e_1 ) \right)  \, dy \right|
$$

$$
\le \frac{1}{|B_\epsilon|} \int_{D_\epsilon} 2 \sqrt{ \epsilon^2 - ||y||^2} \, dy = \frac{|B_\epsilon|}{|B_\epsilon|} = 1 \rimply
$$

$$
| D_\epsilon f| \le 1, \; \forall \; u \in \textrm{Ball}(\reals^n)
$$.


\noindent
Moreover, 

$$
    |f(0) - g(0)| = \frac{1}{|B_\epsilon|}  \left| \int_{D_\epsilon} (g(y) - g(0))\right| \le  \frac{1}{|B_\epsilon|}  \int_{D_\epsilon} |g(x) - g(0)|| \, dy = \frac{ |S^{n-1} |}{|B_\epsilon|} \int_0^\epsilon r^n \, dr  
$$

$$
    = \frac{\epsilon^{n+1} }{n+1} \frac{ |S^{n-1}|}{ \epsilon^n |B_1| } = \frac{\epsilon }{n+1} \frac{ |S^{n-1}|}{  |B_1| }  = \frac{ n \epsilon  }{n+1}
$$

    
%$D_u f(x) = \nabla f(c) \cdot u, ||u|| = 1$ gives $|D_u|$

\end{proof}


Back to concentration, we now know that $f$ concnetrates on a set of large measure, but where? Given an $\epsilon$, and 1-Lipschitz function $f: S^{n-1} \rarw \reals$, then there exists a subspace $V \subset \reals^n$, $\textrm{dim}(V)$ linear in n, s.t. $f_{V \cap S^{n-1}}$ is $\epsilon-$close  to a constant. Note, $S^{n-1}$ is $(n-1)$-sphere, $S^{n-1} \cap V$ is again a unit sphere of dimension $\textrm{dim}(V)-1$.

\begin{thm}

There exists a universal constant, $\kappa >0$, s.t. $\epsilon>0, \forall n \in \nats$, any 1-Lipschitz function $f:S^{n-1} \rarw \reals$ there exists a constant, $C$ (e.g. median or averge) and a subspace $V \subset \reals^n$, s.t. $|f(x) - C| \le \epsilon \forall x \in V \cap S^{n-1}$ and $\textrm{dim}(V) \ge \frac{\kappa \epsilon^2}{\ln(1/ \epsilon)} n$.

\end{thm}

To prove this, we begin with a lemma about equidistributed points on the sphere

\begin{lemma}
Given $n$-dimensional vector space $W$, with norm $|| \cdot ||$ and $ \Sigma = \{ x \in W; ||x|| = 1 \}$, then $\forall \delta > 0, \exists S \subset \Sigma$ with 

1) $\forall x \in S$, $\inf \{  || y-x||; y \in A, y \not = x \} \le \delta$

2) $|S| \le ( 1 + \frac{2}{\delta})^n$

The same is true for appropriate sets in $\{ x \in W; ||x|| =1 \}$.

\end{lemma}




\end{document}

