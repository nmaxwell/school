
\documentclass[landscape,twocolumn,letterpaper]{book}

\usepackage{amsmath}
\usepackage{amssymb}
\usepackage{amsfonts}
\usepackage{latexsym}


\title{Real Analysis Notes}

%\author{Nick Maxwell}

\begin{document}


\chapter{Basics}

\begin{equation*}
\cup_{E \in \mathcal{E} } \, E = \{ x : \, x\in E, \, \textrm{some} \, E \in \mathcal{E}  \}
\end{equation*}

\begin{equation*}
\cap_{E \in \mathcal{E} } \, E = \{ x : \, x\in E, \, \textrm{all} \, E \in \mathcal{E}  \}
\end{equation*}

\begin{equation*}
 \mathcal{E} = \{ E_\alpha : \, \alpha \in A \} = \{E_\alpha \}_{\alpha \in A}
\end{equation*}

\begin{equation*}
\limsup E_n = \cap_{k=1}^{\infty} \cup_{n=k}^{\infty} E_n
\end{equation*}

\begin{equation*}
\liminf E_n = \cup_{k=1}^{\infty} \cap_{n=k}^{\infty} E_n
\end{equation*}

\begin{equation*}
\liminf E_n = \{ x : x \in E_n \, \textrm{for all but finitely many} \, n \}
\end{equation*}

\begin{equation*}
\limsup E_n = \{ x : x \in E_n \, \textrm{for infinitely many} \, n \}
\end{equation*}

\begin{flushleft}
Def: $E_n$ a sequence of sets, $n \in \mathbb{N}$. Take the statement ``$x \in E_n$ for all but finitely many $n$'' to precicely mean ``$\exists k \in \mathbb{N}$ s.t.$ \, x \in \cap_{n=k}^{\infty} E_n $''. Then, `` $x \in E_n$ for infinitely many $n$'' means ``$ x \in \cup_{n=k}^{\infty} E_n,\, \forall k \in \mathbb{N}$''.
\end{flushleft}

\begin{equation*}
x \in \limsup E_n = \cap_{k=1}^{\infty} \cup_{n=k}^{\infty} E_n \Leftrightarrow \left( x \in \cup_{n=k}^{\infty} E_n, \; \forall k \in \mathbb{N} \right)
\end{equation*}

\begin{equation*}
x \in \liminf E_n = \cup_{k=1}^{\infty} \cap_{n=k}^{\infty} E_n \Leftrightarrow \left( x \in \cap_{n=k}^{\infty} E_n, \; \textrm{some} \; k \in \mathbb{N} \right)
\end{equation*}

\chapter{Measure Theory}

\chapter{Probability}













\end{document}









